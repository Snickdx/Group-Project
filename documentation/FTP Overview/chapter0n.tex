%This is the last chapter 
%%=========================================
\chapter[Summary]{Summary and Recommendations for Further Work}

In this final chapter you should sum up what you have done and which results you have got. You should also discuss your findings, and give recommendations for further work.

%%=========================================
\section{Summary and Conclusions}
Here, you present a brief summary of your work and list the main results you have got. You should give comments to each of the objectives in Chapter 1 and state whether or not you have met the objective. If you have not met the objective, you should explain why (e.g., data not available, too difficult).

This section is similar to the Summary and Conclusions in the beginning of your report, but more detailed---referring to the the various sections in the report.

%%=========================================
\section{Discussion}

The application, however, is not without limitations. The application has difficulty transferring large file sizes.

%%=========================================
\section{Recommendations for Further Work}

In addition to the functionality provided, there are several possible improvements that can be made to the application.

\subsection{Short term improvements}
The module for authentication currently used by the server uses a Map hardcoded into the application. As such, it may be desirable to decouple the facilities for authentication from the data. This can be achieved using a database. Integration of a database would involve the writing of an object  that uses the Authenticator trait, and then simply replacing the object being referenced in the AuthenticaionSession class to authenticate users

\subsection{Long term improvements}
The application provides only a subset of the commands that constitute the FTP protocol, giving the application facilities to rename files on the server, retrieval files from the server, delete files on the server, and to send files on the server. As such, to make the system more robust and flexible, more commands specified by the FTP may be implemented 

