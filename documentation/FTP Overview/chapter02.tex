%This is chapter 2
%%=========================================
\chapter{Technical Analysis}
The FTP server has been implemented in Scala. Scala is an objected oriented, functional programming language. Like all JVM languages, Scala compiles to Java Virtual Machine bytecode. Consequently, Scala is interoperable with Java, meaning that code written in Java can be easily used and referenced by Scala code with little or no overhead. 

Scala, being a functional language inspired by members of the ML family languages, has facilities to pattern match on algebraic data types by using their class based representations that implement a special purpose unapply method. There were two primary benefits of using this pattern matching:

\begin{enumerate}
	
	\item Scala's standard library facilitates the conversion of strings representing regular expressions to a objects that implement the unapply method, thereby allowing for the easy dispatch of server activity based on the commands received by the server, as well as the extraction of arguments from the commands without the aide of explicitly defined parsers

	\item Scala's algebraic types allow for the usage of Option types and for the usage of types that resemble Tagged Unions to facilitate dispatch based on the results of computations

\end{enumerate}


The usage of Scala also illustrates that pieces of a Server/Client application can be built using different implementation languages so long as the contracts specified by the underlying protocol of the application is adhered to.

The FTP Client was written in Java. The FTP commands that were implemented will be explored.

%%=========================================
\section{Authentication}



\section{WIP3}


%%=========================================
\section{WIP}

%%=========================================
\section{WIP2}
