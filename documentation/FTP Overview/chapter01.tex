%This is chapter 1
%%=========================================
\chapter{Introduction}
The File Transfer Protocol, hereafter referred to as FTP, is used to
facilitate the transfer of files between computers over a network. It also
allows for the transfer files between FTP servers.
FTP is implemented using a client-server architecture with distinct control
and data connections between the client and the server. This report provides an
overview of an implementation of FTP. A subset of the commands within the protocol
will be defined and examined. This overview also covers the benefits and drawbacks
of the protocol.


%%=========================================
\section{Background}
User groups and organizations rely on FTP to send files directly between computers on a network,
bypassing the need to use physical media such as flash memory or optical discs to transfer said files.
As such, a FTP client/server pair allows a user to carry out file operations with minimal knowledge of the implementation.

\newpage

\section{Objectives}
The main objectives of this Computer Networks project are to
\begin{enumerate}
\item Provide an overview of FTP
\item Implement an FTP Client and Server
\item Facilitate authentication
\item Provide the user with a file explorer
\item Facilitate the upload and download of files
\item Manipulate files on a server
\item Provide functionality to allow a user to mirror folders
\end{enumerate}

%%=========================================
\section{FTP Overview}
Most FTP sessions begin with authentication of user on a given server. Authentication involves the user providing the hostname of the remote host which initiates a TCP
connection to the FTP server in the remote host. The username and password are then sent over the TCP connection via the FTP commands USER and PASS. \par Once authenticated, the user
is then free to carry out file operations based on the level of permissions granted by the Administrator of the given file server. The permissions determine the files and folders
that can be accessed and whether they have read, read/write and access to delete files. FTP facilitates the transfer of files between local and remote file systems.

%%=========================================
\section{FTP Detailed}
FTP runs on TCP and uses two parallel TCP connections to transfer a file, a control connection and a data connection. Sending control information between the two hosts is handled by the control connection.
This control information includes user identification, user password, commands to change the remote directory and also commands to put and get files. All FTP commands are sent via the control
connection. The data connection handles the actual transfer of the file between hosts. Due to the use of a separate control connection, FTP is said to sends its control information out of band.\par
When an FTP session is started, the client initiates a control TCP connection with the server on port 21. The data channel is opened on any unused port set by the application programmer or administrator.
Upon receipt of control commands from the client, the server initiates a TCP data connection to the client side. Only one file is sent per data connection, that is, the file is sent and then the data connection is closed.
The control connection remains open for the duration of the FTP sessions and subsequent files are sent over new data connections that are opened.

There are two ways to send files over the data channel. The first is the Active Mode where the client specifies to the server how the transfer will be done. The client chooses a local port
and mandates that the server send data to that port. A connection on port 20 is initiated by the server and the data is sent. Issues can arise from this setup as it becomes a requirement
for firewalls to allow incoming connections on port 20 to a vast range of ports on on client machines. A vulnerability can be exploited by initiating connections from port 20 in order to
scan internal machines.

The second method of data transfer is Passive Mode. The client requests a file from the server and specifies how the transfer will be done. The server selects an unreserved port
and then increments from that port for each file transferred during that FTP session. The server then notifies the client to connect to the specified port to receive the file. The
upside of this is that since the client initiates the connection then additional ports don't need to be opened in the firewall.

%%=========================================
\section{Structure of the Report}
The rest of the report is structured as follows. Chapter 2 gives an introduction to \ldots
