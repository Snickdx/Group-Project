%This is chapter 1
%%=========================================
\chapter{Introduction}
The File Transfer Protocol, hereafter referred to as FTP, is used to
facilitate the transfer of files between computers over a network. It also
allows for the transfer files between FTP servers.
FTP is implemented using a client-server architecture with distinct control
and data connections between the client and the server. This report provides an
overview of an implementation of FTP. A subset of the commands within the protocol
will be defined and examined. This overview also covers the benefits and drawbacks
of the protocol.


%%=========================================
\section{Background}
User groups and organizations rely on FTP to send files directly between computers on a network,
bypassing the need to use physical media such as flash memory or optical discs to transfer said files.
As such, a FTP client/server pair allows a user to carry out file operations with minimal knowledge of the implementation.

\newpage

\section{Objectives}
The main objectives of this Computer Networks project are to
\begin{enumerate}
\item Provide an overview of FTP
\item Implement an FTP Client and Server
\item Facilitate authentication
\item Provide the user with a file explorer
\item Facilitate the upload and download of files
\item Manipulate files on a server
\item Provide functionality to allow a user to mirror folders
\end{enumerate}

%%=========================================
\section{FTP Overview}
Most FTP sessions begin with authentication of user on a given server. Authentication involves the user providing the hostname of the remote host which initiates a TCP
connection to the FTP server in the remote host. The username and password are then sent over the TCP connection via the FTP commands USER and PASS. \par Once authenticated, the user
is then free to carry out file operations based on the level of permissions granted by the Administrator of the given file server. The permissions determine the files and folders
that can be accessed and whether they have read, read/write and access to delete files. FTP facilitates the transfer of files between local and remote file systems. 

%%=========================================
\section{Comparison to HTTP}
Hypertext Transfer Protocol, HTTP is a file transfer protocol that shares several characteristics with FTP. Both protocols run on TCP with the major difference between the two being that
FTP uses two parallel TCP connections to transfer a file, a control connection and a data connection. 

If there are any ethical problems related to your approach, these should be highlighted and discussed.
%%=========================================
\section{Structure of the Report}
The rest of the report is structured as follows. Chapter 2 gives an introduction to \ldots
